\documentclass[spanish,12pt,a4paper,openany]{book}
\usepackage[T1]{fontenc}
\usepackage[left=2cm, right=2cm, top=2cm, bottom=2cm]{geometry}
\usepackage{graphicx}
\usepackage{mathtools}
\usepackage{amssymb}
\usepackage{amsthm}
\usepackage{fancybox}
\usepackage{thmtools}
\usepackage{nameref}
\usepackage{babel}
\usepackage{hyperref}
%\usepackage{tikz}
%\usetikzlibrary{positioning}
%\tikzset{every picture/.style={line width=0.75pt}} %set default line width to 0.75pt 
\title{[Soluciones] Notas de Teoría de la Medida (L. Rendón)}
\author{ViucheOwO}
\begin{document}
	\begin{titlepage}
		\maketitle
		%\begin{tikzpicture}[x=0.75pt,y=0.75pt,yscale=-1,xscale=1]
		%uncomment if require: \path (0,464); %set diagram left start at 0, and has height of 464
		
		%Shape: Right Triangle [id:dp6483725655468593] 
		%\draw   (483.94,461.77) -- (318.25,257.78) -- (675.39,170.32) -- cycle ;
		%Shape: Right Triangle [id:dp19384348198927914] 
		%\draw   (979.76,461.76) -- (2.64,460.53) -- (522.22,388.8) -- cycle ;
		%Shape: Triangle [id:dp813727602517299] 
		%\draw   (518.39,548.26) -- (389.4,3.17) -- (748.88,346.7) -- cycle ;
		
		%\end{tikzpicture}
	\end{titlepage}
	\tableofcontents
	\chapter{Definiciones preliminares}
	\section{Quiz 1}
	Determine si cada uno de los siguientes enunciados es verdadero o falso:\\
	\begin{enumerate}
		\item  $\mathbb{Q} \in B$ donde $B$ es la $\sigma$ - álgebra de Borel.\\
		\textbf{Verdadero:}
		\begin{proof}
			Note que para cada $ r \in \mathbb{Q}, \left\{ r \right\} \in B$ puesto que este es un cerrado en $\mathbb{R}$, i.e., es complemento de un abierto (véase $\left( -\infty, r \right) \cup \left(r, \infty \right) $ ), que está en $B$. \\
			La enumerabilidad de $\mathbb{Q}$ nos garantiza que $\displaystyle \bigcup_{r \in \mathbb{Q}} \left\{r\right\} = \mathbb{Q} \in B$
		\end{proof}
		\item  $\mathbb{N} \in B$ donde $B$ es la $\sigma$ - álgebra de Borel.\\
		\textbf{Verdadero:}
		\begin{proof}
			La prueba es idéntica a la del punto anterior.
		\end{proof}
		\item El conjunto  $ F = \left\{ M: M \text{ es una } \sigma \text{- álgebra en } \mathbb{R} \right\}$ es enumerable.\\
		\textbf{Falso:}
		\begin{proof}
			Considere la aplicación $\psi: \mathbb{R} \rightarrow F$ tal que $\psi(\alpha) := M_{\alpha}$ donde $M_{\alpha} =: \left\{ \mathbb{R}, \emptyset, (- \infty, \alpha), \left[ \alpha, \infty \right) \right\}$. Considere $\mathcal{M} = Im \psi$, puesto que $\psi$ es inyectiva entonces $|\mathcal{M}| \leq |F|$, por lo tanto $F$ no puede ser contable.
		\end{proof}
		\item La función $f: \mathbb{R} \rightarrow \mathbb{R}$ dada por $f(x) = x^{3}$ es medible, cuando tomamos en $\mathbb{R}$ la $\sigma$ - álgebra de Borel.\\
		\textbf{Verdadero:}
		\begin{proof}
			Probaremos primero por inducción que $f(x)= x^{n}$ es medible para cada $n \in \mathbb{N}$. Para $n=1$ el resultado es obvio puesto que $f^{-1} (V) = V \in B$ para cada $V$ abierto. Supongamos que el enunciado es cierto para $n \geq 1$, luego $f(x) = x^{n+1} = x^{n} \cdot x$ es medible por la proposición 1.1.18. ya que $f = h \circ \varphi$ donde $\varphi : \mathbb{R}^{2} \rightarrow \mathbb{R}$ definida por $\varphi(y,z) = y \cdot z$ es contínua y $h: \mathbb{R} \rightarrow \mathbb{R}^{2} := (x^{n}, x)$ es medible ya que cada una de sus componentes lo es. Por le principio de inducción matemática concluimos que $f$ es medible y tomando el caso $n=3$ tenemos el resultado inicialmente pedido.
		\end{proof}
		\item La función $f: \mathbb{R} \rightarrow \mathbb{R}$ dada por $f(x) = |x|$ es medible, cuando tomamos en $\mathbb{R}$ la $\sigma$ - álgebra de Borel.\\
		\textbf{Verdadero}
		\begin{proof}
			Note que $f(x)=|x| = (h \circ g) (x)$ es medible por la proposición 1.1.17, donde $g(x) = x^{2}$ es medible por el punto anterior y $h(x) = \sqrt{x}$ es contínua.
		\end{proof}
	\end{enumerate}
	\section{Quiz 2}
	Determine si cada uno de los siguientes enunciados es verdadero o falso:
	\begin{enumerate}
		\item  Sean $(X,M)$ espacio meible y $f: X \rightarrow \mathbb{R}.$ si $f$ es medible entonces $|f|$ es medible.\\
		\textbf{Verdadero:}
		\begin{proof}
			Note que $|f| = g \circ f$ donde $g: \mathbb{R} \rightarrow \mathbb{R}$ con  $g(x)=|x|$ es continua. La proposición 1.1.17 garantiza que $|f|$ es medible.
		\end{proof}
		\item Sean $(X,M)$ espacio meible y $f: X \rightarrow \mathbb{R}.$ si $|f|$ es medible entonces $f$ es medible.\\
		\textbf{Falso:}
		\begin{proof}
			Sea $E$ un conjunto no medible y considere $f$ definida como la siguiente función simple: $\chi_{E} - \chi_{E^{c}}$, donde $E^{c} = X-E$
			Es claro que $|f| = 1$ es medible pero $f$ no lo es.
		\end{proof}
		\item Sean $(X,M)$ un espacio medible y $f: X \rightarrow \mathbb{R}$. Si $f$ es medible entonces $f^{+} = sup \left\{ f(x),0 \right\}$ y $f^{-} = sup \left\{-f(x), 0 \right\}$ son medibles.\\
		\textbf{Verdadero}
		\begin{proof}
			Basta probarlo para los abiertos básicos $(\alpha, \beta )$ con $\alpha < \beta$. \\
			Si $\alpha , \beta <0$ entonces $(f^{+})^{-1} ((\alpha, \beta)) = \emptyset$\\
			Si $\alpha, \beta > 0 $ entonces $(f^{+})^{-1} (\alpha, \beta) = f^{-1} ((\alpha, \beta)) \in M$.\\
			Si $\alpha <0, \beta > 0 $ entonces $(f^{+})^{-1} (\alpha, \beta) = (f^{+})^{-1} [0, \beta) = f^{-1} ((-\infty, 0]) \cup f^{-1} ((0, \beta)) \in M$ Note que $f^{-1} ((-\infty, 0]) \in M$ ya que $f^{-1} ((-\infty, 0]) = f^{-1} ((0, \infty)^{c}) = (f^{-1} ((0, \infty)))^{c} \in M$ pues $f^{-1}((0, \infty)) \in M$.
		\end{proof}
		\item $f^{+} = \frac{1}{2} \left(|f| + f \right), f^{+}$ como en el numeral anterior.\\
		\textbf{Verdadero:}
		\begin{proof}
			Si $f(x) \leq 0$ entonces $\frac{1}{2} \left(|f| + f \right) (x) = \frac{1}{2} \left(-f + f \right)(x) = 0$. Ahora, si $f(x)>0$, entonces $\frac{1}{2} \left(|f| + f \right)(x) = \frac{1}{2} \left(f + f \right)(x) = \frac{1}{2}(2f)(x) = f(x)$, lo cual coincide con nuestra definición de $f^{+}$.
		\end{proof}
		\item $f^{-} = \frac{1}{2} \left(|f| - f \right), f^{+}$ como en el numeral anterior\\
		\textbf{Verdadero:}
		\begin{proof}
			La prueba es análoga a la anterior.
		\end{proof}
	\end{enumerate}
	\section{Quiz 3}
	Determine si cada uno de los siguientes enunciados es verdadero o falso:
	\begin{enumerate}
		\item Sea $(X,M)$ un espacio medible entonces toda función simple es medible.\\
		\textbf{Falso:}
		\begin{proof}
			En virtud de la proposición 1.4.3 basta tomar cualquier conjunto no medible $E$ e inmediatamente $\chi _{E}$ es una función simple no medible.
		\end{proof}
		\item $\chi_{A \cup B} = \chi_{A} + \chi_{B}$\\
		\textbf{Falso:}
		\begin{proof}
			Tome $A, B \subset X$ no disyuntos y $ x \in A \cap B$. Note que $\chi_{A \cup B} (x)= 1$, mientras que $\chi_{A} + \chi_{B} (x)= 2$.
		\end{proof}
		\item  $\chi_{A - B} = \chi_{A}(1- \chi_{B})$\\
		\textbf{Verdadero:}
		\begin{proof}
			Si $x \in A-B$, entonces $\chi_{A} (x) =1 , \chi_{B}(x) = 0$, luego $\chi_{A}(1- \chi_{B})(x) = 1(1-0)=1$. Ahora si $x \notin A-B$ entonces $\chi_{A} (x) =0$ o $\chi_{B}(x) = 1$, en ambos casos $\chi_{A}(1- \chi_{B})(x) = 0$.
		\end{proof}
		\item  $\chi_{A \cap B} = \chi_{A}  \chi_{B}$\\
		\textbf{Verdadero}
		\begin{proof}
			Si $ x \in A \cap B,$ entonces $\chi_{A} (x) =1 , \chi_{B}(x) = 1$ y por ende  $\chi_{A}  \chi_{B}(x) = 1$. Ahora, si $x \notin A \cap B$, entonces $\chi_{A} (x) = 0$ , o  $\chi_{B}(x) = 0$ y en ambos casos $\chi_{A}  \chi_{B}(x) = 0$.
		\end{proof}
		\item Sean $(X,M)$ espacio medible y $f \rightarrow [- \infty , \infty]$ una función medible entonces el conjunto $\left\{x \in X : f(x) = \infty \right\}$ es medible.\\
		\textbf{Verdadero:}
		\begin{proof}
			Note que
			\begin{align*}
				\left\{x \in X : f(x) = \infty \right\} &= f^{-1} \left\{ \infty \right\}\\
				&= f^{-1} \displaystyle\left(\bigcap _{n \in \mathbb{N}} [n, \infty ] \right)\\
				&=\displaystyle \bigcap_{n \in \mathbb{N}} f^{-1}([n, \infty])\\
				&= \displaystyle \bigcap_{n \in \mathbb{N}} \left(f^{-1}([-\infty, n)^{c}) \right)\\
				&= \displaystyle \bigcap_{n \in \mathbb{N}} \left(f^{-1}([-\infty, n)) \right)^{c}\\
				&= \displaystyle \left( \bigcup_{n \in \mathbb{N}} f^{-1}([-\infty,n)) \right)^{c}
			\end{align*}
			Donde $f^{-1}([- \infty, n)) \in M$ para cada $n \in \mathbb{N}$, luego la unión contable de estos conjuntos también está en $M$ y por lo tanto su complemento lo está. Asi, $f^{-1} \left\{ \infty \right\} \in M.$
		\end{proof}
	\end{enumerate}
	\section{Ejercicios}
	\begin{enumerate}
		\item Sean $(A_{n})_{n \in \mathbb{N}}$ una sucesión de conjuntos en $X$, muestre que
			\begin{itemize}
				\item $\chi_{\cup_{i=1}^{n} A_{i}} = 1 - \prod_{i=1}^{n}(1- \chi_{A_{i}})$
				\begin{proof}
					Si $ x \in \cup_{i=1}^{n} A_{i} $, entonces $ \chi_{A_{i}} = 1$ para algún $1 \leq i \leq n$, de modo que el factor $(1- \chi_{A_{i}}) = 0$ y por lo tanto todo el producto $ \prod_{i=1}^{n}(1- \chi_{A_{i}}) = 0 $ y así $1 - \prod_{i=1}^{n}(1- \chi_{A_{i}}) = 1 - 0  = 1.$\\
					Por otro lado, si $x \notin \cup_{i=1}^{n} A_{i}$ es porque $x \notin A_{i}$ para todo $1 \leq i \leq n$, esto es, $ \chi_{A_{i}} = 0$, y $\prod_ {i=1}^{n} (1- \chi_{A_{i}} = 1 - \prod_{1}^{n}(1) = 0.$ 		
				\end{proof}
				\item $\chi_{\cap_{i=1}^{n} A_{i}} = \prod_{i=1}^{n}\chi_{A_{i}}$.
				\begin{proof}
					Si $ x \in \cap_{i=1}^{n} A_{i}$ entonces $\chi_{A_{i}} (x) = 1$ para todo $1 \leq i \leq n$, de modo que $\prod_{1}^{n} \chi_{A_{i}} (x) = \prod_{1}^{n} 1 = 1$. \\
					Si $x \notin \cap_{i=1}^{n} A_{i} $ entonces el factor $\chi_{A_{i}}(x)= 0$ para algún $i$, luego $\prod_{1}^{n} = \chi_{A_{i}} (x) = 0$.
				\end{proof}
				\item $\chi_{ \limsup A_{n}} = \limsup\chi_{A_{n}}$.
				\begin{proof}
					\begin{align*}
						\limsup\chi_{A_{n}} (x) &= \limsup \left\{\chi_{A_{n}} (x)\right\} \\
						&= \inf \left\{\sup \left\{\chi_{A_{k}}(x) \right\} : k \geq n \right\}\\
						&= \begin{cases*}
								1 \text{ si } x \in \limsup{A_{n}}\\
								0 \text{ si } x \notin \limsup A_{n} 
							\end{cases*}\\
						&= \chi _{\limsup A_{n}}
					\end{align*}
				\end{proof}
				\item $\chi_{ \liminf A_{n}} = \liminf\chi_{A_{n}}$.
				\begin{proof}
					Indéntica al punto anterior.
				\end{proof}
			\end{itemize}
		\item Sean $(X,M)$ un espacio medible y $\left\{f_{n}: X \rightarrow [- \infty, \infty] \right\}_{n \in \mathbb{N}}$ una sucesión de funciones medibles. Muestre que:
			\begin{itemize}
				\item $\left\{ x \in X : \sup f_{n} (x) \leq a, n \in \mathbb{N} \right\} = \displaystyle \bigcap_{n \in \mathbb{N}} \left\{x \in X : f_{n} (x) \leq a\right\}$
				\begin{proof}
					Sea $x \in \left\{ x \in X : \sup f_{n} (x) \leq a, n \in \mathbb{N} \right\}$. Note que $f_{n}(x) \leq \sup\left\{f_{n} (x) \right\} \leq a$ para todo $n \in \mathbb{N}$, luego $x \in f_{n}^{-1}([-\infty, a])$ para todo $n$, esto es $ x \in \displaystyle \bigcap_{n \in \mathbb{N}} \left\{x \in X : f_{n} (x) \leq a\right\} $.\\
					Reciprocamente, si $x \in \bigcap_{n \in \mathbb{N}} \left\{x \in X : f_{n} (x) \leq a\right\} $ entonces $f_{n}(x) \leq a$ para todo $n \in \mathbb{N}$. Esto significa que $a$ es una cota superior del conjunto $\left\{f_{n}(x)\right\}_{n \in \mathbb{N}}$, por lo que $\sup \left\{f_{n}(x)\right\}_{n \in \mathbb{N}} = \sup f_{n}(x) \leq a$, y en consecuencia $x \in \sup f_{n} ^ {-1} ([-\infty, a]), n \in \mathbb{N} = \left\{ x \in X : \sup \left\{ f_{n} (x) \right\}  \leq a, n \in \mathbb{N} \right\}.$
				\end{proof}
\newpage
		\item $\left\{ x \in X : \inf f_{n} (x) < a, n \in \mathbb{N} \right\} = \displaystyle \bigcup_{n \in \mathbb{N}} \left\{x \in X : f_{n} (x) < a\right\}$.
			\begin{proof}
				Sea $x$ tal que $\inf f_{n} (x) < a$, entonces para algún $k \in \mathbb{N}$ se cumple que $f_{k}(x) < a$ (de lo contrario tendríamos que $a$ es una cota inferior del conjunto $\left\{f_{n}(x)\right\}$ mayor que el $\inf$), luego $x \in \displaystyle \bigcup_{n \in \mathbb{N}} \left\{x \in X :  f_{n}(x) < a\right\}$.\\
				Ahora, si $f_{k}(x) < a$ para algún k, es claro que $\inf f_{n}(x) \leq f_{k}(x) < a$, esto es $x \in \inf f_{n} ^{-1}([-\infty,a))$.
			\end{proof}
		\end{itemize}
		\item Sea $(A_{n})_n \in \mathbb{N}$ un sucesión de conjuntos en $X$.\\
		Tomando $E_{0} = \emptyset$ y $E_{n} = \cup _{k=1 ^{n}} A_{k}, F_{n} = A_{n} - E_{n-1} $ para cada $n \in  \mathbb{N}$.\\
		Muestre que
		\begin{itemize}
			\item 	$E_{n-1} \subseteq E_{n}$ para todo $n \in \mathbb{N}$
			\begin{proof}
				Si $x \in E_{n-1}$, entonces $x \in A_{k}$ para algún $k$ entre $1$ y $n-1$, luego $x \in A_{k} \cup A_{n}$ de modo que $x \in E_{n}$.\\
			\end{proof}
			\item  $F_{i} \cap F_{j} = \emptyset $ si $i \neq j$.
			\begin{proof}
				Supongamos que $i<j$. Si $x \in F_{i}$ y $x \in F_{j}$ entonces $x \in A_{j}$ y $x \notin A_{k}$ para todo $k<j$, lo que contradice $x \in A_{i}$.
			\end{proof}
			\item $\cup _{n=1} ^{\infty} E_{n} = \cup _{n=1} ^{\infty} F_{n} = \cup _{n=1} ^{\infty} A_{n}$
			\begin{proof}
				Es evidente que $\cup _{n=1} ^{\infty} E_{n} = \cup _{n=1} ^{\infty} A_{n}$ por como están definidos los $E_{n}$. Otro hecho que salta a la vista es que $\cup _{n=1} ^ {\infty} F_{n} \subset \cup _{n=1} ^{\infty} A_{n}$ por la definición de los $F_{n}$.\\
				Lo único que queda por notar es que $\cup_{k=1} ^{n} F_{k} = \cup_{k=1} ^{n} A_{k}$ para cada $n \geq 1$, donde nuevamente tenemos una inclusión gratis. Ahora, considere el mayor $k \leq n $ tal que $x \in A_{k}$. Como $A_{k} = (A_{k} \cap E_{k-1}) \cup (A_{k} - E_{k-1}) = (A_{k} \cap E_{k-1}) \cup F_{n} $. Si $x \in F_{n}$ hemos terminado. Si $x \in A_{k} \cap E_{k-1} = \cup_{j=1} ^{k-1} (A_{k} \cap A_{j})$, basta con tomar el mínimo $j$ tal que $x \in A_{k} \cap A_{j}$. Por lo tanto $x \in A_{j} - E_{j-1} = F_{j}$. (Si $x$ perteneciera a $E_{j-1}$, pertenecería a algún $A_{i}$ con $i<j$). Luego $x \in \cup_{k=1 ^{n}} F_{k}$ y concluimos $\cup_{k=1} ^{\infty} F_{k} = \cup_{k=1} ^{\infty} A_{k}$\\			
			\end{proof}
		\end{itemize}
			\item Sea $(A_{n})_n \in \mathbb{N}$ un sucesión de conjuntos en $X$.\\
			
			Defina:\\
			$\limsup A_{n} = \cap _{m=1} ^{\infty} (\cup _{n=m} ^{\infty} A_{n})$.\\ 
			$\liminf A_{n} = \cup _{m=1} ^{\infty} (\cap _{n=m} ^{\infty} A_{n})$.\\
			
			Muestre que:\\
			$\emptyset \subseteq \liminf A_{n} \subseteq \limsup A_{n} \subseteq X$.
			
			\begin{proof}
				La única inclusión no trivial es $\liminf A_{n} \subseteq \limsup A_{n}$.\\
				
				Primero note que $\left\{\cup_{n=k} ^{\infty} A_{n}\right\}_{k \in \mathbb{N}}$ es una secuencia decreciente.\\
				
				Sea $x \in \liminf A_{n}$. Existe $p \in \mathbb{N}$ tal que para todo entero $k \geq p, x \in A_{k}$ y por ende $x \in \cup_{n=k} ^{\infty} A_{n}$, por lo dicho en la línea de arriba, $\cup_{n=k} ^{\infty} A_{n} \supseteq\cup_{n=p} ^{\infty} A_{n}$, para todo $1 \leq k \leq p$, luego $x \in \cup_{n=1} ^{\infty} A_{n}$ para todo entero $n \geq 1$, ergo, $x \in \cap _{m=1} ^{\infty} (\cup _{n=m} ^{\infty} A_{n}) = \limsup A_{n}$.
			\end{proof}
		\item Sea $(A_{n})_n \in \mathbb{N}$ un sucesión de conjuntos en $X$ tal que $A_{i} \subseteq A_{i+1}$ para todo $i \in \mathbb{N}$. Muestre que:\\
		
		$\limsup A_{n} = \cup_{n=1} ^{\infty} A_{n}= \liminf A_{n} $.
		\begin{proof}
			
			 Note que $\cup_{k=m}^{n} A_{k} = \cup_{k=1} ^{n} A_{k}= A_{n}$ y $\cap_{k=m}^{n} A_{k} = A_{m}$ , $m \in \mathbb{N}, m \leq n$ cualquiera sea $n \in \mathbb{N}$. Luego $\cap_{k=m}^{\infty} A_{k} = A_{m}$.\\
			 
			 Ahora, $\liminf A_{n} = \cup _{m=1} ^{\infty} (\cap _{n=m} ^{\infty} A_{n}) = \cup _{m=1} ^{\infty} A_{m}$.\\
			 
			 Por su parte, $\limsup A_{n} = \cap _{m=1} ^{\infty} (\cup _{n=m} ^{\infty} A_{n}) = \cap _{m=1} ^{\infty} (\cup _{n=1} ^{\infty} A_{n}) = \cup_{n=1} ^{\infty} A_{n}$.\\
		\end{proof}
		\item Sea $(A_{n})_n \in \mathbb{N}$ un sucesión de conjuntos en $X$ tal que $A_{i} \supseteq A_{i+1}$ para todo $i \in \mathbb{N}$. Muestre que:\\
		
		$\limsup A_{n} = \cap_{n=1} ^{\infty} A_{n}= \liminf A_{n} $.
		
		\begin{proof}
			La prueba es análoga a la del punto anterior y se deja como ejercicio al lector.
		\end{proof}
		
		\item Sean $(X,M, \mu)$ un espacio de medida y $(A_{n})_{n \in \mathbb{N}}$ una sucesión de conjuntos medibles, Muestre que:\\
		
			\begin{itemize}
				\item  $\mu (\liminf A_{n}) \leq \liminf \mu (A_{n}) $
				
				\begin{proof}
					Note que 
				\end{proof}
				
				\item  $\limsup \mu(A_{n}) \leq \mu (\limsup A_{n}) $, si $\mu(\cup_{n=1} ^{\infty}) < \infty$. 
			\end{itemize}
			
	\end{enumerate}
	\chapter{La medida de Lebesgue}
	\section{Ejercicios}
	\chapter{La integral}
	\section{Ejercicios}
	\chapter{Medida producto}
	\section{Ejercicios}
	\chapter{Espacios Lp}
	\section{Ejercicios}
	\chapter{Algunos tipos de convergencia}
	\section{Ejercicios}
	\chapter{Cargas}
	\section{Ejercicios}
	
\end{document}